%% ------------------------------------------------------------------------- %%
\chapter{Trabalhos Relacionados}
\label{cap:trabalhos-relacionados}

Durante a revisão bibliográfica foram encontrados alguns trabalhos relacionados
ao estudo de dados de movimentaçao de veículos . . .

\section{Dynamic Multiscale Visualization of Flight Data}

-> Resumo Rápido: O artigo traz novas abordagens para a aplicação de técnicas de visualização
para análise de dados do tráfego aéreo. Ele aborda novas fórmulas de se aplicar o bundling,
density maps e animação que outros autores não abordavam. Ele detalha como as técnicas
são aplicadas e que tipo de padrões elas permitem visualizar. Outro ponto importante é
que são afirmados que outros sistemas de controle do tráfego aéreo não mostram as informações
como o sistema proposto (que mostra tanto a posição instantânea como durante o tempo). O seu sistema
possui alguns parâmetros configuráveis que permitem uma exploração dos dados que
trazem vários insights para visualização de padrões e outliers nos dados.

-> Highlights:
  - exploração das informações disponíveis e detecção de outliers
  - visão de grandes áreas do tráfego e durante longos períodos (e.g. dados do mundo durante 1 mês)
  - adaptação de várias técnicas baseadas em imagem (bundling, animation, density maps) para visualizar
  padrões ao longo do tempo
  - visualização com pouca oclusão
  - análise em "tempo real" com processamento na GPU
  - real world datasets

-> Limitações
  - Ainda gera alguma oclusão, como no dataset do mundo todo
  - Software não disponível
  - não implementa queries para filtros (e.g. aviões de altitude maior que X)
  - Suas trilhas apresentam apenas três atributos (velocidade, direção, altura)

\section{Visualizing interchange patterns in massive movement data}

\section{Mapping to Cells: A Simple Method to Extract Traffic Dynamics from Probe Vehicle Data}

