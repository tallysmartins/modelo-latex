%%%%%%%%%%%%%%%%%%%%%%%%%%%%%%%%%%%%%%%%%%%%%%%%%%%%%%%%%%%%%%%%%%%%%%%%%%%%%%%%
%%%%%%%%%%%%%%%%%%%%%%%%%%%%% METADADOS DA TESE %%%%%%%%%%%%%%%%%%%%%%%%%%%%%%%%
%%%%%%%%%%%%%%%%%%%%%%%%%%%%%%%%%%%%%%%%%%%%%%%%%%%%%%%%%%%%%%%%%%%%%%%%%%%%%%%%

% Este pacote define o formato da capa, páginas de rosto, dedicatória e
% resumo. Se você pretende criar essas páginas manualmente, não precisa
% carregar este pacote nem definir os dados abaixo.
\usepackage{imeusp}

% Define o texto da capa e da referência que vai na página do resumo;
% "masc" ou "fem" definem se serão usadas palavras no masculino ou feminino
% (Mestre/Mestra, Doutor/Doutora, candidato/candidata).
\mestrado[masc]
%\doutorado[masc]

% Se "\title" está em inglês, você pode definir o título em português aqui
\tituloport{ElixirBench - uma plataforma para execução de benchmarks para projetos Elixir}

% Se "\title" está em português, você pode definir o título em inglês aqui
\tituloeng{}

% Se o trabalho não tiver subtítulo, basta remover isto.
%\subtitulo{um subtítulo}

% Se isto não for definido, "\subtitulo" é utilizado no lugar
%\subtituloeng{a subtitle}

\orientador[masc]{Dr. Daniel Macêdo Batista}

\programa{Ciência da Computação}

% Se isto não for definido, "\programa" é utilizado no lugar
\programaeng{Computer Science}

\localdefesa{São Paulo}

\datadefesa{14 de Junho de 2018}

% Se isto não for definido, "\datadefesa" é utilizado no lugar
\datadefesaeng{August 10th, 2017}

% Necessário para criar a referência do documento que aparece
% na página do resumo
\ano{2018}

% Palavras-chave separadas por ponto e finalizadas também com ponto.
\palavraschave{benchmark, elixir, software livre}


\direitos{Autorizo a reprodução e divulgação total ou parcial
deste trabalho, por qualquer meio convencional ou
eletrônico, para fins de estudo e pesquisa, desde que
citada a fonte.}
