\section{Como submeter suas mudanças?}
\label{cap:como-submeter-suas-mudancas}

 A gestão do projeto é feita primordialmente pelo repositório no Github, onde
 desenvolvedores podem submeter suas mudanças através de \textit{Pull Requests}.
 Idealmente, um \textit{Pull Requests} adiciona uma mudança coesa ao código, e
 geralmente condensada em apenas um \textit{commit}, que descreve o que foi
 alterado.

 Cada repositório possui diferentes responsáveis que avaliam as mudanças submetidas.
 Por exemplo, não necessariamente o maior contribuidor do repositório \textit{backend}
 irá aceitar uma alteração no repositório do \textit{frontent}. Com isso, é necessário
 conhecer as pessoas certas para comunicação e interação no repositório, conforme
 o objetivo da mudança.

 O Github oferece uma boa documentação de como criar um \textit{Pull Request},
 como pode ser visto no link (em inglês): \url{https://help.github.com/articles/creating-a-pull-request}
