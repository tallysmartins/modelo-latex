\section{Como configurar seu ambiente de trabalho?}
\label{cap:como-configurar-seu-ambiente}

 Para configurar o ambiente de trabalho é recomendado a utilização de sistema
 operacional baseado em GNU Linux, onde as dependências são facilmente configuradas.
 Como o projeto é dividido em três principais sub repositórios, back-end, front-end e
 runner, a configuração de cada um pode ser feita isoladamente seguindo as especificações
 do README.

 No entanto, algumas especificações de versões que não são explicitamente
 informadas no repositório tiveram que ser "adivinhadas". Para trabalhar no
 período da disciplina, foram utilizadas as pacotes padrão da distribuição
 Debian Jessie, que não apresentaram nenhum problema de compatibilidade com o projeto:

 \textbf{Backend}: Elixir 1.6, PostgreSQL 9.6

 \textbf{Frontend}: Node v8.10.0, React 16.2.0

 \textbf{Runner}: Elixir 1.6, Docker 18.0 + Docker-Compose 1.8

 O ecossistema Elixir não é como o do Java, ou R, que possuem grandes IDEs, então
 o desenvolvedor deve buscar um editor de texto de sua preferência para adentrar
 no código. 
