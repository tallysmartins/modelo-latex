\chapter{Guia Rápido}
\label{cap:guia-rapido}

\noindent
 \textbf{Nome do projeto:} ElixirBench

\noindent
 \textbf{Descrição:} Serviço para execução de testes como TravisCI ou GitlabCI, voltado
 para testes de desempenho.

\noindent
 \textbf{Conhecimentos Técnicos:} Elixir, React and Docker

\noindent
 \textbf{Repositório:} \url{https://github.com/elixir-bench}

\noindent
 \textbf{Recursos para ajudar:}
 \begin{itemize}
   \item \href{https://github.com/PragTob/benchee}{Biblioteca de benchmarks Benchee}
   \item \href{https://github.com/elixir-ecto/ecto/tree/mm/benches/bench}{Exemplos de benchmarks escritos para a biblioteca Ecto}
   \item \href{https://www.howtographql.com/graphql-elixir/0-introduction}{Introdução ao GraphQL}
 \end{itemize}

\noindent
 \textbf{Como começar:} Se inscreva nos canais de comunicação e tente executar
 o projeto a partir do README.md de cada repositório. Após isso, certifique
 que os testes estão rodando e passando.

\noindent
 \textbf{Pré-requisitos para configurar o ambiente:}
 \begin{itemize}
   \item distribuição GNU Linux
   \item Elixir 1.6, PostgreSQL 9.6
   \item Node v8.10.0, React 16.2.0
   \item Docker 18.0 + Docker-Compose 1.8
 \end{itemize}

\noindent
 \textbf{Como submeter suas mudanças:} Envie um Pull Request no GitHub

\noindent
 \textbf{Comunicação:}
 \begin{itemize}
   \item \underline{irc-channel:}  \#beam-community on irc.freenode.net
   \item \underline{mailing-list:} groups.google.com/forum/\#!forum/beam-community
 \end{itemize}

\noindent
 \textbf{Licença:} Apache V.2.
