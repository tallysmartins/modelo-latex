\section{Como lidar com o código?}
\label{cap:como-lidar-com-codigo}

 O fato de o projeto ser dividio em várias frentes abre um
 leque maior de possibilidades. Desenvolvedores com maior afinidade
 por desenvolvimento de interfaces, html, css e javascript podem começar o
 contato pelo repositório do frontend, por exemplo, já usuários com mais
 afinidade em backend, rotas, controllers, modelos, podem explorar primeiro
 o repositório do servidor.

 O projeto é uma aplicação web simples, com rotas para cadastro e recuperação
 de objetos. A maior complexidade é entender a relação entre esses objetos. Para
 compreende-los melhor, o desenvolvedor pode recorrer a leitura do 
 arquivo \textbf{priv/repo/seeds.exs} que contém dados exemplo para popular
 o banco de dados. Olhando esse arquivo é possível
 ter uma visão melhor de como são criados os objetos e como eles se relacionam.
 Além disso, a suíte de testes automatizados do projeto descrevem os vários
 cenários e apresentam de forma mais clara a lógica e restrições por trás do código.

 Por outro lado, existe uma complexidade maior ao tentar entender o código do
 Runner que escuta por novos Jobs, executa-os em containers Docker e retorna
 os resultados para o servidor. Existem alguns detalhes que requerem um maior
 conhecimento de como o Docker funciona, seus comandos e suas flags. Este definitivamente
 não é o local ideal para começar a explorar e contribuir para o projeto.


