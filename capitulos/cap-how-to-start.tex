\section{Como começar?}
\label{cap:como-comecar}

 Para iniciar no projeto, o primeiro passo é ter um conhecimento básico de aplicações
 web e programação Elixir e React, já que este é o ambiente geral da ferramenta. O projeto
 está dividido em quatro sub-repositórios, e é necessário um certo nível de compreensão
 de cada um deles para iniciar. É possível que um contribuidor consiga fazer
 colaborações isoladas para cada uma das partes porém, para contribuições mais complexas é necessário
 ter todo o ambiente configurado.  O projeto se divide principalmente em:

 \textbf{Backend}: Aplicação Elixir construida com framework Phoenix (similar ao rails) que
 fornece uma API para consulta dos dados dos repositórios cadastrados, jobs executados
 e resultados dos jobs.

 \textbf{Frontend}: Aplicação React que consome a API do servidor e apresenta
 a página HTML iterativa para exploração dos dados.

 \textbf{Runner}: É uma espécie de worker. Uma aplicação Elixir que consome a API
 do servidor e fica encarregada de "ouvir" novos Jobs cadastrados, executá-los
 e retornar seus resultados.

 Para nenhum dos repositórios existe ainda uma lista de tarefas publicamente definida, ou
 um mapeamento de issues, por isso é necessário que as pessoas entrem em contato com os mantenedores
 por algum dos canais de comunicação (ver Cap. \ref{cap:como-se-comunicar}) para
 manifestarem sua intenção em contribuir e discutirem como o podem fazer.

 Usuários que gostam de investigar o código e executar localmente para conhecer
 e entender o que está por trás não devem ter muitos problemas para executar
 o projeto com as instruções dadas no README.md, na página do Github de cada
 repositório. No entanto, precisam entender o básico de aplicações web Elixir
 e React para, no mínimo, instalarem as dependências e executarem o software
 localmente com todos os testes passando.
