\chapter{Introdução}
\label{cap:introducao}

Esse capítulo apresenta de maneira geral aspectos e características da plataforma
ElixirBench. Abordaremos aspectos técnicos, como linguagem, licença, funcionalidades
e também aspectos não técnicos, como história, utilização e outros.

O projeto ElixirBench, escolhido para a disciplina, é uma ferramenta
de integração contínua, como Travis ou GitlabCI, porém destinada para a execução
de \textit{builds} de desempenho (\textit{benchmarks}) e disponível apenas para projetos escritos na
linguagem de programação Elixir. A ideia geral é ter a execução dos testes
 de

The chosen project is ElixirBench, a tool for monitoring the quality of packages written in the Elixir programming language and the language itself. It is an initiative of the Elixir community and is still under development. The tool is also subscribed to the Google Summer of Code event and the project description can be found in this wiki page(idea #3). The general idea is to get packages benchmarks running on each new commit or new version released. So, the work of the ElixirBench tool is to automatically trigger, execute and display the results of the benchmarks in a web system. This flow is similar to the execution of automated tests in continuous integration tools, such as Travis or Jenkins.

\section{Objetivos}
\section{Histórico}
\section{Principais Funcionalidades}
\section{Utilização ao redor do mundo}
